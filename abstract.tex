\chapter*{Abstract}

Providing accessible products has recently left  to be a differential
feature of certain companies. Accessibility, today, is a fundamental requirement
of any developed solution, indicating primarily respect and care to
customers.
This statement is especially true for products designed to the Internet which is
the gateway of all  world intercommunication. The Internet has showed to be the
fastest and cheapest technology to acquire information, and has forced legacy
technologies (banking services, for example) to adapt itself so that people with
permanent or momentary difficulties can be able to interact with society.
However, to give an accessible product is not always an easy task. In addition to several different classes of disabilities / difficulties (which leads to
different accessibility problems), lack of training and experience in the area
makes developers producing code in a wrong way, resulting in an inaccessible
product.
The process models and software development frameworks have not been adapted in a
consistent and homogeneous way, contemplating the accessibility in the software
factory. We are going through a transition phase between from the HTML and
XHTML 4 to HTML 5, which among other things, aims to deliver a semantic web and
to treat specific problems of accessibility, but it's not yet fully
consolidated.
Finally, the tools available to developers cannot effectively assist developers to
deliver an affordable product. In this work, we intend to address the problem of
the limited support on development of accessible softwares which is a real issue
in avaliable software development tools, by building a plugin for Eclipse, a
highly customizable and extensible IDE (Integrated Development Environment) used
by many developers across the world.

\label{abstract}