\chapter{Dificuldades Enfrentadas}

Encontramos certa dificuldade na hora de modelar as estruturas de dados para representar o problema de forma efici�nte.\\
\noindent Resolvemos por utilizar a estrutura de dados \textit{List}, que faz parte da biblioteca padr�o do C++, por ter uma implementa��o otimizada para casos gerais, e ser de f�cil manipula��o ( possui interface rica, com m�todos iteradores e m�todos para inser��o e remo��o de elementos )\\
Tivemos algumas dificuldades na utiliza��o da estrutura List, o que demandou pesquisa, mas acarretou em um bom aprendizado sobre a \textit{STL} ( Standard Template Library ) e a linguagem C++.\\
\noindent Ainda na STL, tivemos alguns problemas na hora de utilizar a fun��o \textit{min\_element}, que foi necess�rio no nosso m�todo Prim::extraiMinimo. Para utiliza-la, tivemos de sobrecarregar os operadores de compara��o da classe NodeEdge.\\