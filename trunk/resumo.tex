\chapter*{Resumo}

Fornecer produtos acess�veis deixou de ser um diferencial de
determinadas empresas. Acessibilidade, nos dias atuais, � um requisito fundamental de qualquer solu��o desenvolvida, indicando principalmente
respeito e cumplicidade com os clientes. Essa afirma��o � especialmente
verdadeira para os produtos desenvolvidos para a \textit{Internet}, porta de
acesso para toda a intercomunica��o mundial. A \textit{Internet} se mostrou a
tecnologia mais r�pida e barata de aquisi��o de informa��o, levando tecnologias
legadas (servi�os banc�rios, por exemplo) a se adaptarem de forma que pessoas
com dificuldades permanentes ou moment�neas consigam interagir com a sociedade.
Contudo, fornecer um produto acess�vel nem sempre � uma tarefa f�cil. Al�m de
diversas classes diferentes de defici�ncias e dificuldades (o que acarreta
problemas de acessibilidade diferentes), a falta de treinamento e experi�ncia na
�rea faz com que desenvolvedores cometam erros em v�rios aspectos, resultando
num produto inacess�vel. Os modelos de processos e \textit{frameworks} de desenvolvimento de
\textit{software} ainda n�o se adaptaram de forma consistente e homog�nea,
em rela��o a acessibilidade na f�brica de \textit{software}. A �rea de
Tecnologia da Informa��o est� passando por uma fase de transi��o entre o
\textit{HTML 4 e XHTML} para o \textit{HTML 5}, que, entre outras coisas,
pretende enfatizar a \textit{web} sem�ntica e tratar dos problemas
espec�ficos de acessibilidade. Por fim, as ferramentas dispon�veis aos desenvolvedores
n�o conseguem, de maneira eficaz, auxiliar efetivamente os desenvolvedores a
entregarem um produto acess�vel. \textbf{Neste trabalho, pretende-se contribuir
fornecendo suporte ao desenvolvimento de \textit{softwares}
acess�veis, construindo um \textit{plugin} para o \textit{Eclipse}, uma
\textit{IDE (Integrated Development Environment)} altamente customiz�vel e
extens�vel utilizada por muitos desenvolvedores em todo o mundo}.

\label{resumo}